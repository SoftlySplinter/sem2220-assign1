\documentclass[10pt, a4paper]{article}

\usepackage{amsmath}
\usepackage{amssymb}
\usepackage{graphicx}
\usepackage{listings}
\usepackage{color}
\usepackage[section]{placeins}
\usepackage{paralist}
\usepackage{fullpage}

\usepackage{caption}
\usepackage{subcaption}

\definecolor{mygreen}{rgb}{0,0.6,0}
\definecolor{mygray}{rgb}{0.5,0.5,0.5}
\definecolor{mymauve}{rgb}{0.58,0,0.82}

\lstdefinelanguage{JavaScript}{
  keywords={typeof, new, true, false, catch, function, return, null, catch,
switch, var, if, in, while, do, else, case, break},
  keywordstyle=\color{blue}\bfseries,
  ndkeywords={class, export, boolean, throw, implements, import, this},
  ndkeywordstyle=\color{darkgray}\bfseries,
  identifierstyle=\color{black},
  sensitive=false,
  comment=[l]{//},
  morecomment=[s]{/*}{*/},
%  commentstyle=\color{purple}\ttfamily,
%  stringstyle=\color{red}\ttfamily,
  morestring=[b]',
  morestring=[b]"
}

\lstset{ %
  backgroundcolor=\color{white},
  basicstyle=\footnotesize,
  breakatwhitespace=false,
  breaklines=true,
  captionpos=b,
  commentstyle=\color{mygreen},
  escapeinside={\%*}{*)},
  extendedchars=true,
  keepspaces=true,
  keywordstyle=\color{blue},
  rulecolor=\color{black},
  showspaces=false,
  showstringspaces=false,
  showtabs=false,
  stepnumber=2,
  stringstyle=\color{mymauve},
  tabsize=2,
}

\newcommand*{\titleGM}{\begingroup
\hbox{ 
\hspace*{0.2\textwidth} 
\rule{1pt}{\textheight} 
\hspace*{0.05\textwidth} 
\parbox[b]{0.75\textwidth}{ 

{\noindent\Huge\bfseries Mobile Web}\\[2\baselineskip] % Title
{\large \textit{SEM2220 - Assignment 1}}\\[4\baselineskip] % Tagline or further description
{\Large \textsc{Alexander D Brown (adb9)}} % Author name

\vspace{0.5\textheight} 
}}
\endgroup}


\title{Mobile Web}
\author{Alexander D Brown (adb9)}

\begin{document}
\titleGM 
\tableofcontents
\newpage

\section{Introduction}
This report shows the process undertaken to change the Conference Web
Application (produced by Chris Loftus) from a statically generated list to one
loaded dynamically using JavaScript from a Web SQL
database\cite{WebSQL2010Hickson}.

The Conference Web Application is a website which was designed using
progressive enhancement to provide content to mobile devices as well as desktop
browsers. It uses the jQuery Mobile framework\cite{jQueryMobile} to give a 
native feel for mobile users. 

The process involved the implementation of some JavaScript code to query the
Web SQL database created in HTML 5 storage when the Conference Web Application is
initially loaded. The results of this query are then dynamically rendered in
the sessions page as a part of a jQuery Mobile list view.



\section{Implementation}
This section describes the process taken to implement the dynamic loading of
sessions into the session view from the pre-existing Web SQL database. There
were two main steps to do this:

\begin{enumerate}
\item Query the database using standard SQL statements.
\item Render the results of the query.
\end{enumerate}

Because a lot of the code was already written this was a simple case of
implementing two methods. The method to query the database handled the building
of SQL to perform the query and use a callback method to handle the results of
this query. This callback method also needed implemented and read the each of
the results and appended them to a jQuery Mobile list view object using jQuery
methods.

\subsection{Implementing the Database Query}
The \texttt{DataContext.js} file handles the data handling for the application,
on the first load of the whole site the database is built from hard-coded
values in the file. It is also responsible for performing database queries and
transactions and therefore is the location which the sessions query is
required.

To start with a simple SQL statement was designed, based on the hints from the
original author which would select everything from the sessions table where the
session was on the first day of the conference.

\begin{lstlisting}[language=sql]
SELECT * FROM sessions WHERE sessions.day_id = '1'
                       ORDER BY sessions.starttime ASC
\end{lstlisting}

From this it is a simple matter of calling a method from the transaction method
as shown:

\begin{lstlisting}[language=javascript]
var querySessions = function(tx) {
  var sql = // SQL Statement
  tx.execute_sql(sql, [], callback, error_callback);
};
\end{lstlisting}

Where the callback is the rendering function.

To make this statement a little more secure, as recommended by the API, it was
changed to be:

\begin{lstlisting}[language=javascript]
var querySessions = function(tx) {
  var sql = "SELECT * FROM sessions WHERE session.day_id = ? 
             ORDER BY sessions.starttime ASC";
  tx.execute_sql(sql, [1], callback, error_callback);
};
\end{lstlisting}

Though the query is completely isolated from user input, it may not be in the
future.

To enhance this further, joining the days table to include the name of the day
that the session was on. This could have been hard coded in the rendering
function, but again it may not remain limited to the first day so doing this
dynamically will be a benefit in the future.

\begin{lstlisting}[language=javascript]
var querySessions = function(tx) {
  var sql = "SELECT sessions.* days.day FROM sessions, days 
             WHERE sessions.day_id = ? 
             AND days._id = sessions.day_id 
             ORDER BY sessions.dayid 
             AND sessions.starttime ASC";
  tx.execute_sql(sql, [1], callback, error_callback);
};
\end{lstlisting}

The improve the callback functionality, a helper method was used to provide 
only the rows from the results set to the callback as the other two attributes 
are based on non-select SQL statements.

\subsection{Implementing the Rendering Function}
The \texttt{Controller.js} file handles a lot of the application processing,
particularly dynamic jQuery Mobile elements, but also with some geolocation
processing. It is therefore the home of the function to render sessions onto
the correct page.

The rendering function acts as a callback to the aforementioned database query,
and receives a Web SQL \texttt{SQLResultSetRowList} from the query. The 
rendering function then enumerates over each row. From this information a 
series of HTML can be generated and rendered on the page.

First off, using the jQuery API, the element with the correct ID is found on 
the page and bound to a variable. From this raw HTML was appended to this
element, with each individuals attributes inserted at the correct points.
Finally, the bound element is initialised as a jQuery Mobile list view and
refreshed using the API.

To improve this, several helper functions were written; one to convert the
resulting rows from the \texttt{SQLResultSetRowList} to a standard JavaScript
array to allow easier iteration. Such a function would be of use in extending
the application, so it was well worth implementing.

A second function to render an individual session was also implemented, so that
the processing could be abstracted away. This function returned a list item 
HTML element which could then easily be appended to the correct list in the
main function.

\subsection{Problems Encountered}
The main issues found with this assignment related to the Web SQL database. The
specification is very terse and, as such, doesn't contain a lot of information
as to how the methods and objects actually work.

Because of the nature of JavaScript it can be quite difficult to debug, so
working out the problems with the database connection could be quite difficult
on occasion.

The main issue I ran into is there's no formal specification of the SQL
statements Web SQL supports. When writing more advanced queries, particularly
those involving joins between two or more tables, was a matter of constantly
running the query to see if it would work then changing the statement until
some form of correct result was gained.

%\section{Preview}
% TODO Screenshots if there's room

\section{Testing and Debugging}
% TODO Screenshots of the developer view, winre and emulator testing.


\section{Evaluation}


\newpage
\bibliographystyle{plain}
\bibliography{citations}

\end{document}
